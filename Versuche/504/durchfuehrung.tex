\subsection{Erstellung einer Kennlinienschaar}
Das regelbare Konstantspannungsgerät mit dem maximalen Strom von 2,6A wird zur Erzeugung der Heizspannung genutzt. Für die Anodenspannung wird ein Gerät genutzt, das bis zu 250V Spannung erzeugen kann. Der Pluspol wird mit der Anode Verbunden und der Minuspol mit dem bereits an der Kathode angeschlossen Minuspol der Heizspannung. Zum Ablesen der jeweiligen Spannung und Stromstärke können die eingebauten Volt- und Amperemeter verwendet werden.

Anschließend wird für einen festen Heizstrom die Heizspannung gemessen und für Anodenspannungen von 0V bis 250V in 10V Schritten der Anodenstrom gemessen.
Die selbe Messung wird anschließend für drei weite Heizströme und den maximal möglichen als fünfte Messung durchgeführt.

\subsection{Anlaufstrom}
Um den Anlaufstrom zu messen wird für die Anodenspannung ein Konstantspannungsgerät mit einer variablen Spannung von 0V bis 1V verwendet. Es wird genau anders herum gepolt angeschlossen als für die Erstellung der Kennlinien, um ein Gegenfeld zu erzeugen. Dass heißt, der Pluspol des Spannungsgeräts wird an den Minuspol der Heizspannung geschlossen. Weiterhin ist es notwendig auch ein empfindlicheres Amperemeter zu verwenden, das im Nanoampere-Breich messen kann. Der Minuspol des Spannungserzeugers wird an den LO-Eingang des Amperemeters angeschlossen. Aufgrund der Kontakt- und Leitungswiderstände, ist es wicht ein möglichst Kurzes Kabel zwischen HI-Ausgang und Anode zu verwende. Der Kontaktwiderstand lässt sich verringern indem man den Bananenstecker in der Buchse dreht.