% \subsection{Kennlinien der Hochvakuumdiode}
	% Unter Verwendung des Literaturwertes\cite{tafel} von
% \begin{align}
	% e_0\varphi&=4{,}53eV
 % \end{align}
 % lassen sich nach der Richardson-Gleichung (Gl.\ref{eqrichard}) die Sättigungsstromgrenzen berechnen (Tab.\ref{tablitis}).
 % \begin{table}[h]
	\begin{center}
		\begin{tabular}{cccc}
			T[kK]&$I_{S,mess}[mA]$&$I_{S,errechnet}[A]$&$\Delta I_S$ [\%] \\ \hline
			2,0&0,16&10&100 \\
			2,1&0,38&42&100 \\
			2,1&0,72&79&100
		\end{tabular}
		\caption{Vergleich von $I_{S,mess}$ und $I_{S,errechnet}$}
		\label{tablitis}
	\end{center}
\end{table}
 % Vergleicht man nun $I_{S,mess}$ mit $I_{S,errechnet}$, so sind große Unterschiede zu erkennen.\\
 % Das könnte daran liegen, dass schon relativ kleine Unterschiede der Ausgangswerten $T$ und $I_S$
 % das Ergebnis stark beeinflussen.
 % \FloatBarrier
\subsection{Langmuir-Schottkysches-Gesetz}
Der Theoriewert des Exponenten der Strom-Spannungs-Beziehung lässt sich aus Gleichung \ref{eqlangmuir} ablesen.
Der Exponent der Messwerte wurde in Gleichung \ref{exalpha} bestimmt.
\begin{align}
Theorie: I&\sim U^{\alpha_{errechnet}}=U^{\frac{3}{2}}=U^{1{,}5} \\
Messung: I&\sim U^{\alpha_{mess}}=U^{1{,}18} \\
\Delta \alpha &=22\%
\end{align}
Dieser Unterschied bedeuted, wie auch in Abbildung \ref{picb1} zu erkennen ist, dass die Stromstärke
mit steigender Spannung nicht so schnell zu nimmt, wie sie es theoretisch sollte. \\
Dieser Differenz könnten Stromstärkeverluste an zu Grunde liegen, die daher kommen, dass auf Grund eines
Hochvakuums geringer Qualität zu viele Elektronen an Luftmolekülen wechselwirken.
% \subsection{Kathodentemperatur im Anlaufstromgebiet}
% Bei der Untersuchung des Anlaufstromgebietes in Bezug auf die Kathodentemperatur sind folgende Ergebnisse
% gemessen worden (vgl. \ref{tabc1})
% % identisch zu: tabc1
\begin{table}[h]
	\begin{center}
		\begin{tabular}{cc}
			$6{,}1V-U_{Heiz,real}$[$\mu$V]&2298K-T[$\mu$K] \\ \hline
			0,11&207 \\
			0,091&205\\
			0,065&202\\
			0,045&200\\
			0,030&199\\
			0,020&198\\
			0,013&197\\
			0,0075&197\\
			0,0045&196\\
			0,0029&196\\
			0,0017&196
		\end{tabular}
		\caption{Kathodentemperatur im Anlaufstromgebiet($U_{Heiz}=6,1V$)}
		\label{tabanlauft}
	\end{center}
\end{table}
% Da die Heizspannung entgegen dem Konstantspannungsgerät und parallel zur Diode gepolt ist, wirkt sich der Spannungsabfall längs des
% Heizfadens nicht entscheident auf das Messergebnis aus.
% \FloatBarrier
\subsection{Kathodentemperatur bei Saugspannung}
Die Messergebnisse der Kathodentemperatur sind in Tabelle \ref{tabkattemp} zusammengefasst (vgl. Tab. \ref{tabd1}).
% identisch zu tabd1
\begin{table}[h]
	\begin{center}
		\begin{tabular}{ccc}
			$U_{Heiz}$[V]&$I_{Heiz}$[A]&T[kK] \\ \hline
			% 10	&0,044\\
			4{,}2&2{,}1&2{,}0\\
			4{,}8&2{,}2&2{,}1\\
			5{,}0&2{,}3&2{,}1\\
			5{,}9&2{,}5&2{,}3\\
			6{,}1&2{,}6&2{,}3
		\end{tabular}
		\caption{Kathodentemperatur unter Heizleistungsvariation}
		\label{tabkattemp}
	\end{center}
\end{table}
\FloatBarrier
\subsection{Austrittsarbeit des Kathodenmaterials}
Die errechnete Austrittsarbeit (Gl.\ref{eqphi}) unterscheidet sich signifikant von dem Literaturwert\cite{tafel}.
\begin{align}
	e_0\varphi_{Lit}&=4{,}53eV \\
	e_0\varphi_{mess}&=6{,}5eV\\
	\Delta e_0\varphi&=43\%
 \end{align}
 Ein Grund für diesen Unterschied könnte sein, dass die reale Temperatur geringer war, als die aus den Messergebnissen errechnete.
 
