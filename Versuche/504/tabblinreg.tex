\begin{table}[h]
	\begin{center}
		\begin{tabular}{cccc}
			U[V] & I[mA] & ln(U) & ln(I)\\ \hline
			10	&0,063& 2,30& -2,765\\                  
			20	&0,170& 2,99& -1,772\\
			30	&0,288& 3,40& -1,245\\
			40	&0,398& 3,68& -0,921\\
			50	&0,514& 3,91& -0,666\\
			60	&0,606& 4,09& -0,501\\
			70	&0,711& 4,24& -0,341\\
			80	&0,829& 4,38& -0,188\\
			90	&0,958& 4,49& -0,043\\
			100	&1,107& 4,60& 0,102\\
			110	&1,282& 4,70& 0,248\\
			120	&1,399& 4,78& 0,336\\
			130	&1,494& 4,86& 0,402\\
			140	&1,593& 4,94& 0,466\\
			150	&1,730& 5,01& 0,548\\
			160	&1,865& 5,07& 0,623\\
			170	&2,000& 5,13& 0,693\\
			180	&2,140& 5,19& 0,761\\
			190	&2,290& 5,24& 0,829\\
		\end{tabular}
		\caption{Logarithmen zur Kennlinie 5 (Heizwerte: 6,1V; 2,6A)}
		\label{tabblinreg}
	\end{center}
\end{table}