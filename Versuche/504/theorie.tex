Die Atome in Metallen sind fast ausnahmslos ionisiert. Das führt dazu, dass die Atome sich zu gitterartigen Strukturen zusammenfügen und die zugehörigen Elektronen sich fast frei innerhalb des Gitters bewegen können. Besitzen die Elektronen genug Energie, können sie aus dem Metall austreten. Als Modell, kann das Potential des Metalls als Potentialtopf betrachtet werden. Die Arbeit, die ein Elektron aufbringen muss um das Metall zu Verlassen heißt Austrittsarbeit.
Wie viele Elektronen einen gewissen Betrag an Energie haben, lässt sich durch die Fermi-Diracsche Verteilungs-Funktion beschreiben.
\begin{align}
f(E)=\frac{1}{e^\frac{E-\zeta}{kT}+1}
\end{align}
Dabei steht $\zeta$ für die Fermische Grenzenergie, die in diesem Fall die Austrittsarbeit ist.
\\ 
Bei dem in diesem Versuch betutzten Wolfram ist die Exponentialfunktion selbst beim Schmelzpunkt so groß, dass die Gleichung vereinfacht werden kann.
 
\begin{align}
f(E)=\frac{1}{e^\frac{E-\zeta}{kT}}
\end{align}
\\
Aus Gleichung (2) lässt sich auch auf die Richardson-Gleichung schliessen.
\begin{align}
j_s(T)=4\pi\frac{e_0 m_0 k^2}{h^3}T^2e^\frac{-e_0\phi}{kT}
\end{align}
Die Sättigungsstromgröße $j_s$ gibt and wie viele Elektronen pro Zeit und Fläche aus dem Metall austreten.
\\
Für die Messung in diesem Versuch wird eine Hochvakuum-Diode benutzt. In ihr befindet sich ein Draht aus Wolfram, der mit einer Heizspannung $U_f$ zum glühen gebracht wird. Der Draht dient dabei als Kathode, von der die ausgelösten Elektronen zur Anode wandern. Ohne ein Vakuum, würden die Elektronen mit den Gasmolekülen in Wechselwirkung treten und die Anode nicht erreichen.
\\
Die Strom-Spannungs Kurve der Diode setzt sich aus drei Teilen zusammen:
\\
Das Anlaufstromgebiet, ist das Gebiet in dem Strom durch die Diode fließt, ohne dass eine Anodenspannung angelegt wird. Sogar bei einer kleinen Gegenspannung $(U<=-1V)$ fließt noch ein schwacher Strom. Dies kommt durch die Eigengeschwindigkeit der Elektronen zustande, da zumindes einige laut Gleichung (1) genug Energie besitzen um das Metall zu verlassen.
\\
Der zweite Teil der Kurve, das Raumladungsgebiet, entsteht aufgrund des erzeugten elektrischen Feldes zwischen Anode und Kathode. Es erreichen nicht alle Elektronen, die das Metall verlassen, auch die Anode, da sie zwar genug Energie besitzen  um aus dem Metall auszutreten, aber nicht genug um die Anode zu erreichen.
Beschrieben wird die Dichte durch die Langmuir-Schottkysche Raumladungsgleichung.
\begin{align}
I=\frac49e_0\sqrt{2\frac{e_0}{m_0}}\frac{V^\frac32}{a^2} \label{eqlangmuir}
\end{align}
Dabei ist $V$ die Anodenspannung und $a$ der Abstand zwischen Anode und Kathode.
\\
Der letzte Teil Teil ist das Sättigungsgebiet. Hier nähert sich der Strom asymptotisch einem Maximalwert, dem Sättigungswert. \\

% Ergänzt
Eine weitere Gleichung (\ref{eqtemp}) lässt sich aus dem Energiesatz und dem Stefan-Boltzmannschen Gesetz \cite{anleitung}
herleiten. Damit kann die Kathodentemperatur berechnet werden.
\begin{align}
I_Heiz*U_Heiz&=f*\eta*\sigma*T^4+N_WL \label{eqtemp}
\end{align}
Wobei $f$ die emittierende Kathodenoberfläche angibt, $\eta$ der Emissionsgrad der Oberfläche ist, $\sigma$ 
die Stefan-Boltzmannsche Strahlungskonstante meint und $N_WL$ die Wärmeleitung der Apparatur bezeichnet.




