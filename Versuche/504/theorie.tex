Die Atome in Metallen sind fast ausnahmslos ionisiert. Das führt dazu, dass die Atome sich zu gitterartigen Strukturen zusammenfügen und die zugehörigen Elektronen sich fast frei innerhalb des Gitters bewegen können. Besitzen die Elektronen genug Energie, können sie aus dem Metall austreten. Als Modell, kann das Potential des Metalls als Potentialtopf betrachtet werden. Die Arbeit, die ein Elektron aufbringen muss um das Metall zu Verlassen heißt Austrittsarbeit.
Wie viele Elektronen einen gewissen Betrag an Energie haben, lässt sich durch die Fermi-Diracsche Verteilungs-Funktion beschreiben.
\begin{align}
f(E)=\frac{1}{e^\frac{E-\zeta}{kT}+1}
\end{align}
Dabei steht $\zeta$ für die Fermische Grenzenergie, die in diesem Fall die Austrittsarbeit ist.
\\ 
Bei dem in diesem Versuch betutzten Wolfram ist die Exponentialfunktion selbst beim Schmelzpunkt so groß, dass die Gleichung vereinfacht werden kann.
 
\begin{align}
f(E)=\frac{1}{e^\frac{E-\zeta}{kT}}
\end{align}
\\
Aus Gleichung (2) lässt sich auch auf die Richardson-Gleichung schliessen.
\begin{align}
j_s(T)=4\pi\frac{e_0 m_0 k^2}{h^3}T^2e^\frac{-e_0\phi}{kT}
\end{align}
Die Sättigungsstromgröße $j_s$ gibt and wie viele Elektronen pro Zeit und Fläche aus dem Metall austreten.
\\
Für die Messung in diesem Versuch wird eine Hochvakuum-Diode benutzt. In ihr befindet sich ein Draht aus Wolfram, der mit einer Heizspannung $U_f$ zum glühen gebracht wird. Der Draht dient dabei als Kathode, von der die ausgelösten Elektronen zur Anode wandern. Ohne ein Vakuum, würden die Elektronen mit den Gasmolekülen in Wechselwirkung treten und die Anode nicht erreichen.
\\
Aufgrund des erzeugten elektrischen Feldes zwischen Anode und Kathode entsteht eine Raumladungsdichte. Denn, es erreichen nicht alle Elektronen, die das Metall verlassen, auch die Anode, da sie zwar genug Energie besitzen können um aus dem Metall auszutreten, aber nicht genug um die Anode zu erreichen.
Beschrieben wird die Dichte durch die Langmuir-Schottkysche Raumladungsgleichung.
\begin{align}
j=\frac49e_0\sqrt{2\frac{e_0}{m_0}}\frac{V^\frac32}{a^2}
\end{align}

