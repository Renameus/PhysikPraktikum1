\subsection{Eigenschaftenbestimmung der Versuchsanordnung}
Um die Winkelrichtgröße $D$ des Versuchsapperates zu bestimmen,
wurde in einem Abstand $r$ von der Drehachse die Kraft $F(\phi)$, 
welche senkrecht zur Drehachse und zum Drehachsenabstand wirkte, 
abhängig von dem Auslenkwinkel gemessen.\\
Das Eigenträgheitsmoment $I_D$ der Drillachse wurde durch die Messung  
unterschiedliche Schwingungsdauer $T$ bei verschiedenen Abständen
zweier Massen von der Drehachse bestimmt, welche sich auf einer 
leichten Metallstange senkrecht zur Drehachse befanden. Auch wurden die
Gewichte der Masssen und der Stange bestimmt.
\subsection{Trägheitsmoment einer Kugel und einer Scheibe} \label{subsec:KugelScheibe}
Die Holzkugel, beziehungsweise die Holzscheibe, wurde in die Versuchsanordnung
eingespannt, sodass die Drehachse durch den Massenschwerpunkt führte.
Daraufhin wurde das Objekt un einen kleinen Drehwinkel ausgelenkt und die jeweilige
Schwingungsdauer gemessen. Außerdem wurden die Massen und Ausmaße der Körper bestimmt.
 \subsection{Trägheitsmoment einer Holzpuppe} 
Die Versuchsdurchführung verlief analog zu \ref{subsec:KugelScheibe}.
Dabei wurde die Holzpuppe in zwei physikalisch verschiedenen Stellungen aufgestellt, einmal
mit senkrecht zur Drehachse ausgestreckten Armen und Beinen, das zweite mal mit  
senkrecht zur Drehachse ausgestreckten Beinen und angewinkelten Armen.
Eine Versuchsanordnung mit angewinkelten Beinen war aufgrund einer aperiodischen Schwingung
nicht möglich. Desweiteren wurden bei der Körpermaßbestimmung mehrere Messwerte an verschieden
Stellen des Körpers zwecks einer späteren Vereinfachung durch Zylinder gemessen.
