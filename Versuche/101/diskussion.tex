\subsection{Eigenträgheitsmoment und Winkelrichtgröße}
Aus der Messung ergab sich die Winkelrichtgröße
$D=(5,642\pm1,247)*10^{-4}\frac{Fr}{^{\circ}}$


Das Eigenträgheitsmoment
$I_{D}=(-2,833\pm 0,013)*10^{-3}kg*m^2$
ist kleiner Null und wurde daher bei den nachfolgend Berechnungen vernachlässigt. Ein negativer Wert tritt auf, da das Trägheitsmoment des Stab, im Vergleich zur Drillachse, ein sehr hohes Trägheitsmoment hat. Kleine ungenauigkeiten bei der Messung haben eine große Auswirkung auf das Ergebins. Des weiteren gillt Gleichung \ref{eqti} nur für kleine Winkel, bei denen eine genause Messung der Periodendauer nicht möglich ist.
\\
\\
\subsection{Trägheitsmoment der Kugel und Scheibe}
Aus der Messung wurden die Trägheitsmomente
\begin{align*}
I_{Scheibe}=4,700*10{-5}kg*m^2\\
I_{Kugel}=4,739*10^{-5}kg*m^2
\end{align*}
bestimmt.


Die Berechnung nach Gleichung \ref{eqkugel} und \ref{eqzylsenk} ergab
\begin{align*}
I_{Scheibe}=2,532*10^{-3}\text{kg*m$^2$}\\
I_{Kugel}=6,944*10^{-2}\text{kg*m$^2$}
\end{align*}

\subsection{Trägheitsmoment der Trägheitspuppe}
Aus Gleichung \ref{eqkugel} bis \ref{eqzylpar}
\begin{align*}
I_{an}=(1,289\pm0,279)*10^{-3}\text{kg*m$^2$}\\
I_{an}=(1,397\pm0,280)*10^{-3}\text{kg*m$^2$}
\end{align*}
\\
\\
Die Berechnung aus der gemessenen Periodendauer ergibt Trägheitsmomente von
\begin{align*}
I_{an}=0,708*10^{-6}\text{kg*m2}\pm23,01\%\\
I_{ab}=1,112*10^{-6}\text{kg*m2}\pm22,14\%
\end{align*}
\\
\\
\subsection{Abweichung der gemessenen von den errechneten Werten}
Ins gesamt sind alle durch die Messung der Periodendauer ermittelten Werte $10{-3}$ mal kleiner als die errechneten. Das lässt darauf schließen, dass die ermittelte Winkelrichtgröße zu klein ist.
Fehler können durch ungenaues ablesen auf der lose aufgelegten Messskala oder ungenaue Kraftmesser entstanden sein.