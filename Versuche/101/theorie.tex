Rotationsbewegungen lassen sich durch drei Parameter charakterisieren.
Zum einen durch das Trägheitsmoment $I$, zum anderen durch das Drehmoment $M$ und
die Winkelbeschleuningung $\ddot \phi$.\\
Das Trägheitsmoment ist von der Drehachsenlage und von der Masse des drehenden Körpers abhängig.
 \begin{align}
	\text{Punktmasse } I&=m r^2 \label{eqpunktmasse}\\
	\text{i Punktmassen } I&=\sum_{k=1}^i m_k r_k^2 \\
	\text{infinitisimale Massen } I&=\int r^2 dm \label{eqinfmasse}\\
	\text{mit } r: \text{Drehachsenabstand} \nonumber
 \end{align}
 Aus diesen Gleichungen (Gl.\ref{eqpunktmasse}-\ref{eqinfmasse}) lassen sich Trägheitsmomente
 für geometrische Standartkörper mit Drehungen um ihren Masseschwerpunkt herleiten.
 \begin{align}
	\text{langer dünner Stab}(\vec l \perp \vec \omega) \text{  } I&=\frac{1}{12} m l^2 \\
	\text{Kugel } I&=\frac{2}{5} m R^2 \label{eqkugel}\\
	\text{Zylinder} (\vec R \perp \vec \omega) \text{  } I&=\frac{1}{2} m R^2 \label{eqzylsenk}\\
	\text{Zylinder} (\vec R \| \vec \omega) \text{  } I&=m (\frac{1}{4} R^2 + \frac{1}{12} h^2) \label{eqzylpar}
 \end{align}
 Für den Fall, dass die Schwerpunktachse um $a$ parallel verschoben ist, gilt der Steinersche Satz:
 \begin{align}
	I&=I_{Schwerpunkt} + m a^2
	\label{eqsteiner}
 \end{align}
 Das Drehmoment $M$ bei die Rotation ist vergleichbar mit der Kraft $F$ bei Translation.
 \begin{align}
	\vec M &= \vec F \times \vec r \\
	M &= F  r\text{  },\text{  }\vec r \perp \vec F\\
	M&=D \phi\\
	\Rightarrow D&=\frac{F r}{\phi} \text{  },\text{  } \vec r \perp \vec F \label{eqdstat}\\
	D: \text{Winkelrichtgröße} \nonumber
 \end{align}
 Das Direktionsmoment $D$ beschreibt also eine Proportionalitätskonstante zwischen der
 Auslenkung $\phi$ und $M$. Dieser Zusammenhang kann also eine durch Torsion hervorgerufene
 rücktreibende Kraft beschreiben, durch welche das System Schwingungen ausführen kann.
 Dabei ist eine weitere Größe die Schwingungsdauer $T$. Damit gibt es nun zwei Möglichkeiten
 zur Bestimmung des Direktionsmoments. Die statistische Methode (Gl. \ref{eqdstat}) folgt aus dem 
 Zusammenhang mit dem Drehmoment. Die dynamische Methode (Gl. \ref{eqddyn}) lässt sich bei bekanntem
 Trägheitsmoment aus der Periodendauer $T$ bestimmen.
\begin{align}
	T&=2\pi\sqrt{\frac{I}{D}} \text{ (für kleine Winkel)} \label{eqti}\\
	\Rightarrow D&=I(\frac{T}{2\pi})^{-2} \label{eqddyn}
\end{align}
