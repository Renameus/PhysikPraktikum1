%
%
\subsection{Fourier-Synthese}
Zuerst wird mit einem Oszilloskop der Oberwellensynthesizer auf die richtige 
Phasenlage kalibriert. Dazu wird die Grundschwingung auf den X-Eingang und die 
einzustellende Schwingung auf den Y-Eingang gelegt. Dann wird die dabei enstehende
Lissajous-Figur soweit durch verändern der Phase verändert, bis eine einzelne Kurve
zu sehen ist. Dann kann durch Phasensprünge von $90^{\circ}$ oder $180^{\circ}$ die Phase
zu den Oberwellen passend eingestellt werden.\\
Jetzt werden die vorher berechneten Amplituden der Fourierreihenkomponenten mit dem
Synthesizer angepasst und dann schrittweise unter Phasenlagenüberprüfung addiert und
auf einem Oszilloskop ausgegeben. Das ganze wird je für eine Dreieck-, eine Sägezahn- und
eine Rechteckspannung durchgeführt. 
\subsection{Fourier-Analyse}
Ein passend eingestellter Funktionengenerator wird an ein Oszilloskop mit eingebautem Rechner
angeschlossen, welcher eine Fourieranalyse (vgl. Gl. \ref{eqfout2}) durchführt und auf dem Bildschirm
ausgibt. Dann werden mit dem Cursor die Amplituden der einzelnen Schwingungsbestandteile abgelesen.