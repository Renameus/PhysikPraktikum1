%
%
\subsection{Fourier-Synthese}
Zuerst wird mit einem Oszilloskop der Oberwellensynthesizer auf die richtige 
Phasenlage kalibriert. Dazu wird die Grundschwingung auf den X-Eingang und die 
einzustellende Schwingung auf den Y-Eingang gelegt. Dann wird die dabei enstehende
Lissajous-Figur soweit durch ver�ndern der Phase ver�ndert, bis eine einzelne Kurve
zu sehen ist. Dann kann durch Phasenspr�nge von $90\circ$ oder $180\circ$ die Phase
zu den Oberwellen passend eingestellt werden.\\
Jetzt werden die vorher berechneten Amplituden der Fourierreihenkomponenten mit dem
Synthesizer angepasst und dann schrittweise unter Phasenlagen�berpr�fung addiert und
auf einem Oszilloskop ausgegeben. Das ganze wird je f�r eine Dreieck-, eine S�gezahn- und
eine Rechteckspannung durchgef�hrt. 
\subsection{Fourier-Analyse}
Ein passend eingestellter Funktionengenerator wird an ein Oszilloskop mit eingebautem Rechner
angeschlossen, welcher eine Fourieranalyse (vgl. Gl. \ref{eqfout2}) durchf�hrt und auf dem Bildschirm
ausgibt. Dann werden mit dem Cursor die Amplituden der einzelnen Schwingungsbestandteile abgelesen.