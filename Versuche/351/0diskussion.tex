%
%
\subsection{Fourier Synthese}
Während sich die Phaseneinstellung über die Lissajous-Figuren noch relativ
gut durchführen ließ, war die Einstellung der Koeffizienten aufgrund der schwankenden
Größe der Amplituden des Synthezizers schwieriger und damit fehleranfällig.\\
Auf den Abbildungen \ref{picfsr} bis \ref{picfss} ist gut die Näherung der Fourier-Reihe 
an die jeweilige Funktion zu erkennen. Zwar sind die Koeffizienten bis $n=9$ noch eine recht
grobe Näherung, dennoch lassen sich qualitative Unterschiede erkennen.\\ Die Rechteckspannung ist
nur relativ ungenau angenähert, an den Stellen der Unstetigkeit lässt sich gut der Über- beziehungsweise
Unterschwung der Schwingung, also das Gibbsche Phänomen erkennen. Desweiteren ist eine Tendenz 
zu einem abnehmenden Plateau anstatt einem Plateau mit konstanten Wert zu erkennen, was
eventuell mit kleinen Phasendifferenzen zu erklären wäre. Das Plateau selbst ist
auch nicht als Gerade, sondern deutlich als Schwingung genähert.\\
Bei der Dreieckspannung ist hingegen eine gute Annäherung zu erkennen, ohne Sprungstellen
der anzunähernden Funktion, sind nur die abgerundeten Spitzen der Fourier-Reihe als erkennbarer
Unterschied identifizierbar.\\
Die Sägezahnspannung ist ähnlich zur Rechteckspannung zwar zu erkennen, aber nur ungenau
genähert. Auch hier sind die ansteigenden Geraden noch deutlich als Schwingung zu erkennnen und
auch das Gibbsche Phänomen tritt wieder auf.
\subsection{Fourier-Analyse}
Die ausgegebenen Schwingungen des Funktionengenerators passen grundsätzlich
gut zu den erwarteten Werten. Bei allen drei Funktionen pendelt die Genauigkeit
jedoch recht stark, es treten messbare Abweichungen im Bereich von ungefähr
$1\%$ bis $20\%$ auf. Dabei lassen sich jedoch keine Regelmäßigkeiten erkennen,
auch wenn ein Koeffizient ungenau ist, so kann ein nachfolgender Koeffizient wieder
genau sein, was natürlich zur Orthogonalität der Sinus- und Kosinus-Funktionen passt.
Die Abweichungen ließen sich durch Messfehler erklären, denn die Amplituden der Ausgabe
auf dem Bildschirm besaßen einen schwankenden Wert, so dass sich die Amplitude häufig nicht
genau ablesen ließ. 