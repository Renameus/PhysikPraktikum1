\begin{table}[h]
	\begin{center}
		\begin{tabular}{cc}
			U[V]&$\Delta Q [\text{G}\frac{\text{C}}{e}]$ \\ \hline
			339&63,68\\
			342&11,50\\
			345&13,53\\
			348&12,40\\
			351&13,25\\
			354&12,39\\
			357&11,92\\
			360&12,01\\
			410&18,17\\
			460&24,10\\
			510&30,09\\
			560&35,97\\
			610&41,55\\
			650&47,73\\
			653&48,29\\
			656&46,66\\
			659&48,33\\
			662&46,36\\
			665&52,44\\
			668&52,20\\
			671&52,79\\
			674&53,75\\
			677&50,56\\
			680&51,58\\
			683&50,60\\
			686&54,91\\
			689&49,79\\
			692&48,88\\
			695&50,28\\
			698&46,15\\
			701&52,62
		\end{tabular}
		\caption{freigesetzte Ladungsmenge im Zählrohr}
		\label{tabd1copy}
	\end{center}
\end{table}