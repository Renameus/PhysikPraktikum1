Aus Gleichung \ref{eqalinrega} ergibt sich eine Plateau-Steigung von $a=0,5\% \text{ pro } 
100 \text{ Volt}$ mit einem Fehler von $\Delta a=38,0\%$, für einen 290 Volt langen
Plateau-Bereich von 360 Volt bis 650 Volt. Die Messwerte außerhalb des Plateau-Bereichs streuen relativ 
stark in Gegensatz zu dem Plateau-Bereich selbst (vgl. \ref{picA}), in welchem sie aber immer noch um $38\%$
von der Linearisierung abweichen. \\
Ein Ablesen des Oszilloskops ergibt für den geschätzten zeitlichen Abstand zwischen Primär- und
Nachladungsimpulsen einen Wert von $55\text{ } \mu\text{s}$. Die Methode des Ablesens selbst war recht ungenau
und somit schlecht als Messwert brauchbar.\\
Ebenfalls ablesen lässt sich die Totzeit $T_{\text{Oszilloskop}}=40\text{ }\mu\text{s}$. Das Ableseverfahren am Oszilloskop ist allredings ziemlich ungenau, was  auch die Differenz zu der errechneten Totzeit erklärt. \\ Die Totzeit nach
der Zwei-Quellen-Methode ergibt sich nach Gleichung \ref{eqtot2}: $T_{2-\text{Quellen}}=234,5\mu\text{s}$ 
mit $\Delta T_{2-\text{Quellen}}=0,017\%$. Der Unterschied zwischen $T_{\text{Oszilloskop}}$ und $T_{2-\text{Quellen}}$
von $83\%$ zeigt die Problematik der Ablese-Messmethode auf.\\
Die im Zählrohr freigesetzte Ladung (Tab. \ref{tabd1}) folgt aus Gleichung \ref{ladungsmenge}. 
Die Messwerte außerhalb des linearen Bereichs weichen recht groß ab im Gegensatz zu dem linearen Bereich 
selbst (vgl. \ref{picD}), in welchem sie nur noch um $\Delta a=1,3\%$ von der Linearisierung abweichen,
also recht zutreffend sind.  
% \begin{table}[h]
	\begin{center}
		\begin{tabular}{cc}
			U[V]&$\Delta Q [\text{G}\frac{\text{C}}{e}]$ \\ \hline
			339&63,68\\
			342&11,50\\
			345&13,53\\
			348&12,40\\
			351&13,25\\
			354&12,39\\
			357&11,92\\
			360&12,01\\
			410&18,17\\
			460&24,10\\
			510&30,09\\
			560&35,97\\
			610&41,55\\
			650&47,73\\
			653&48,29\\
			656&46,66\\
			659&48,33\\
			662&46,36\\
			665&52,44\\
			668&52,20\\
			671&52,79\\
			674&53,75\\
			677&50,56\\
			680&51,58\\
			683&50,60\\
			686&54,91\\
			689&49,79\\
			692&48,88\\
			695&50,28\\
			698&46,15\\
			701&52,62
		\end{tabular}
		\caption{freigesetzte Ladungsmenge im Zählrohr}
		\label{tabd1copy}
	\end{center}
\end{table}
\FloatBarrier