\subsection{Zählrohr-Charakteristik}
Die Zählrohr-Charakteristik ist in Abbildung \ref{picA}, welche 
die Messdaten aus Tabelle \ref{taba1} verwendet, abgebildet. 
\begin{table}[h]
	\begin{center}
		\begin{tabular}{ccc}
			Messprobe & U/mV & Temperatur T$_\text{i}$/$^{\circ}\mathrm{C}$\\ \hline
			Wasser im Dewar-Gefäß bei Ausgangstemperatur, $T_x$& 0,7 & 17,52\\
			Mischwasser im Dewar-Gefäß bei Endtemperatur, $T_m$&1,9&47,11\\
			erhitzes Wasser, $T_y$&3,3&80,95
			% 10	&0{,}044\\
		\end{tabular}
		\caption{Messwerte und Temperaturen des Versuchsablaufs (Kalorimeter)}
		\label{taba1}
	\end{center}
\end{table} 	\begin{figure}[h]
		\begin{center}
		\includegraphics[scale=0.75]{picA.jpg}
		\caption{Grafisches Auftragen der Messwerte zur Zählrohr-Charakteristik}
		\label{picA}
		\end{center}	
	\end{figure}
\FloatBarrier
Die Länge des Plateau-Bereiches lässt sich darauf aufbauend auf ungefähr
290 Volt, von 360 Volt bis 650 Volt, abschätzen. Dieser Bereich ist in 
Abbildung \ref{picAlinregwitherrors} mit einem Messfehler von unter $1\%$ (\cite{anleitung}, Seite 226) 
noch einmal dargestellt, wobei in diesem Bereich (vgl. \ref{taba1})eine lineare 
Ausgleichsrechnung \cite{linreg} (Gl. \ref{eqlinrega}) programmiert in Python durchgeführt wurde:
\begin{align}
y&=a*x+b \\
\Leftrightarrow N&=0,0050\frac{1}{\text{s V}}*U+101,4606\frac{1}{\text{s}} \label{eqlinrega} \\
a&=0,0050\frac{1}{\text{s V}} \label{eqalinrega}\\
\Delta a&=38,0\% \\
b&=101,4606\frac{1}{\text{s}}\\
\Delta b&=0,9\%
\end{align}
	\begin{figure}[h]
		\begin{center}
		\includegraphics[scale=0.75]{picAlinregwitherrors.jpg}
		\caption{Plateau-Bereich der Zählrohr-Charakteristik}
		\label{picAlinregwitherrors}
		\end{center}	
	\end{figure}
\FloatBarrier
Aus der Linearen Regression lässt sich dann die Plateau-Steigung von $0,5\% \text{ pro } 
100 \text{ Volt }$(Gl. \ref{eqalinrega}) ablesen.
\subsection{Totzeitbestimmung nach der Zwei-Quellen-Methode}
\FloatBarrier
\begin{table}[h]
	\begin{center}
		\begin{tabular}{ccc}
			$N_1$[1/s]&$N_2$[1/s]&$N_{1+2}$[1/s] \\ \hline
			234,26&12,74&245,60\\
		\end{tabular}
		\caption{Zwei-Quellen-Methode}
		\label{tabc2}
	\end{center}
\end{table}
Die Totzeit lässt sich nach Gleichung \ref{2} aus der Zwei-Quellen-Methode mit Tabelle \ref{tabc2} berechnen,
der Fehler ergibt sich aus Gleichung \ref{totzeit2fehler}.
\begin{align}
\frac{N_{1+2}}{1 - T N_{1+2}}&=\frac{N_1}{1 - T N_1} + \frac{N_2}{1 - T N_2} \label{totzeit2fehler}\\
T= 234,5 \mu\text{s} \label{eqtot2}\\
\Delta T&=0,017\%
\end{align}
\subsection{Pro Teilchen freigesetzte Ladungsmenge im Zählrohr}
\FloatBarrier
Die pro Teilchen freigesetzte Ladungsmenge im Zählrohr lässt sich nach Gleichung \ref{ladungsmenge}
berechnen.
Daraus ergeben sich die Ladungsmengen in Tabelle \ref{tabd1} beziehungsweise Abbildung \ref{picD}.
In dem linearen Bereich, abgeschätzt von U$=360$V bis U$=650$V,wurde eine lineare Ausgleichsrechnung \cite{linreg} (Gl. \ref{eqlinregd}) 
programmiert in Python durchgeführt:
\begin{align}
y&=a*x+b \\
\Leftrightarrow Q/e&=(1,2093 * 10^8) \frac{1}{\text{V}} *U+ (-3,1548 * 10^10) \label{eqlinregd} \\
a&=(1,2093 * 10^8) \frac{1}{\text{V}} \label{eqalinregd}\\
\Delta a&=1,3\% \\
b&=-3,1548 * 10^10\\
\Delta b&=2,8\%
\end{align}
\begin{table}[h]
	\begin{center}
		\begin{tabular}{ccc}
			$U_{Heiz}$[V]&$I_{Heiz}$[A]&T[kK] \\ \hline
			% 10	&0,044\\
			4{,}200&2{,}100&1{,}959\\
			4{,}800&2{,}200&2{,}059\\
			5{,}000&2{,}300&2{,}108\\
			5{,}900&2{,}500&2{,}254\\
			6{,}100&2{,}600&2{,}298
		\end{tabular}
		\caption{Kathodentemperatur unter Heizleistungsvariation}
		\label{tabd1}
	\end{center}
\end{table} 	\begin{figure}[h]
		\begin{center}
		\includegraphics[scale=0.75]{picD.jpg}
		\caption{freigesetzte Ladungsmenge im Zählrohr}
		\label{picD}
		\end{center}	
	\end{figure}
\FloatBarrier