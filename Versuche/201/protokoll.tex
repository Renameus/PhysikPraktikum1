% http://fachschaft.physik.uni-dortmund.de/images/GlobaltutAP/protokoll.txt
% ========================================
%	Header einbinden
% ========================================

\input{apheaderneu.tex}
\renewcommand*\rmdefault{iwona}\normalfont\upshape

% ========================================
%	Angaben für das Titelblatt
% ========================================

\title{Versuch 201 - Das Dulong-Petitsche Gesetz\\				% Titel des Versuchs 
\large TU Dortmund, Fakultät Physik\\ 
\normalsize Anfänger-Praktikum}

\author{Marc Posorske\\			% Name Praktikumspartner A
{\small \href{marc.posorske@tu-dortmund.de}{marc.posorske@tu-dortmund.de}}	% Erzeugt interaktiven einen Link
\and						% um einen weiteren Author hinzuzfügen
Fabian Lehmann\\					% Name Praktikumspartner B
{\small \href{fabian.lehmann@tu-dortmund.de}{fabian.lehmann@tu-dortmund.de}}		% Erzeugt interaktiven einen Link
}
\date{29. November 2012}				% Das Datum der Versuchsdurchführung

% ========================================
%	Das Dokument beginnt
% ========================================

\begin{document}

% ========================================
%	Titelblatt erzeugen
% ========================================

\maketitle					% Jetzt wird die Titelseite erzeugt
\thispagestyle{empty} 				% Weder Kopfzeile noch Fußzeile

% ========================================
%	Der Vorspann
% ========================================

%\newpage					% Wenn Verzeichnisse auf einer neuen Seite beginnen sollen
%\pagestyle{empty}				% Weder Kopf- noch Fußzeile für Verzeichnisse

\tableofcontents

%\newpage					% eine neue Seite
%\thispagestyle{empty}				% Weder Kopf- noch Fußzeile für Verzeichnisse
%\listoffigures					% Abbildungsverzeichnis

%\newpage					% eine neue Seite
%\thispagestyle{empty}				% Weder Kopf- noch Fußzeile für Verzeichnisse
%\listoftables					% Tabellenverzeichnis
\newpage					% eine neue Seite


% ========================================
%	Kapitel
% ========================================

%\section{Einleitung}				% Bei Bedarf
%	Das Geiger-Müller-Zählrohr ist ein Instrument um die Intenstät ionisierender Strahlung zu messen. Seine Funktionsweise basiert darauf, dass wenn $\alpha$-, $\beta$- und $\gamma$-Strahlung in seinem Innern absorbiert wird ein elektrischer Impuls erzeugt wird. Es wird auch heute noch aufgrund seiner einfachen Funktionsweise häufig genutzt.
\FloatBarrier
\section{Theorie}
	Die Atome in Metallen sind fast ausnahmslos ionisiert. Das führt dazu, dass die Atome sich zu gitterartigen Strukturen zusammenfügen und die zugehörigen Elektronen sich fast frei innerhalb des Gitters bewegen können. Besitzen die Elektronen genug Energie, können sie aus dem Metall austreten. Als Modell, kann das Potential des Metalls als Potentialtopf betrachtet werden. Die Arbeit, die ein Elektron aufbringen muss um das Metall zu Verlassen heißt Austrittsarbeit.
Wie viele Elektronen einen gewissen Betrag an Energie haben, lässt sich durch die Fermi-Diracsche Verteilungs-Funktion beschreiben.
\begin{align}
f(E)=\frac{1}{e^\frac{E-\zeta}{kT}+1}
\end{align}
Dabei steht $\zeta$ für die Fermische Grenzenergie, die in diesem Fall die Austrittsarbeit ist.
\\ 
Bei dem in diesem Versuch betutzten Wolfram ist die Exponentialfunktion selbst beim Schmelzpunkt so groß, dass die Gleichung vereinfacht werden kann.
 
\begin{align}
f(E)=\frac{1}{e^\frac{E-\zeta}{kT}}
\end{align}
\\
Aus Gleichung (2) lässt sich auch auf die Richardson-Gleichung schliessen.
\begin{align}
j_s(T)=4\pi\frac{e_0 m_0 k^2}{h^3}T^2e^\frac{-e_0\phi}{kT}
\end{align}
Die Sättigungsstromgröße $j_s$ gibt and wie viele Elektronen pro Zeit und Fläche aus dem Metall austreten.
\\
Für die Messung in diesem Versuch wird eine Hochvakuum-Diode benutzt. In ihr befindet sich ein Draht aus Wolfram, der mit einer Heizspannung $U_f$ zum glühen gebracht wird. Der Draht dient dabei als Kathode, von der die ausgelösten Elektronen zur Anode wandern. Ohne ein Vakuum, würden die Elektronen mit den Gasmolekülen in Wechselwirkung treten und die Anode nicht erreichen.
\\
Aufgrund des erzeugten elektrischen Feldes zwischen Anode und Kathode entsteht eine Raumladungsdichte. Denn, es erreichen nicht alle Elektronen, die das Metall verlassen, auch die Anode, da sie zwar genug Energie besitzen können um aus dem Metall auszutreten, aber nicht genug um die Anode zu erreichen.
Beschrieben wird die Dichte durch die Langmuir-Schottkysche Raumladungsgleichung.
\begin{align}
j=\frac49e_0\sqrt{2\frac{e_0}{m_0}}\frac{V^\frac32}{a^2}
\end{align}


\section{Durchführung}
	\subsection{Bestimmung der Wärmekapazität des Kaloriemeters}
Das Dewargefäß wurde leer und halb gefüllt mit destilliertem Wasser gewogen. Die andere Hälfte des Wassers wurde 
ebenfalls gewogen und dann über einer Kochplatte erhitzt. Die temperaturdifferenzabhängigen Spannungen des 
Wassers im Dewar-Gefäß und des erhitzen Wassers im Gegensatz zu einem mit Eiswasser gefüllten 
Behältnisses wurden mit einem Thermoelement bestimmt und direkt im Anschluss ist das erhitze Wasser
in das Dewar-Gefäß gefüllt worden. Nach einer kurzen Zeit des Vermischens wurde abermals die 
temperaturdifferenzabhängige Spannung mit dem Thermoelement gemessen.
\subsection{Bestimmung der Wärmekapazität der Proben}  
Bei den drei Proben von Blei, Graphit und Kupfer wurde jeweils analog vorgegangen, wobei Blei drei
mal den Versuch durchlief. Zuerst wurde die Masse der Proben bestimmt, die daraufhin in Wasser über 
der Kochplatte erhitzt worden sind.
Das Dewar-Gefäß wurde mit frischem destillierten Wasser befüllt gewogen, dann wurde die 
temperaturdifferenzabhängige Spannung des Wassers und der Probe gemessen. Die Probe wurde direkt 
im Anschluss komplett in das Wasser getaucht. Dann ist die temperaturdifferenzabhängige Spannung 
im Wasser und an der Probe solange durchgemessen worden, bis der gleiche Wert auftrat, 
welcher als mischtemperaturdifferenzabhängige Spannung aufgezeichnet wurde.
\section{Auswertung}
	\subsection{Zählrohr-Charakteristik}
Die Zählrohr-Charakteristik ist in Abbildung \ref{picA}, welche 
die Messdaten aus Tabelle \ref{taba1} verwendet, abgebildet. 
\begin{table}[h]
	\begin{center}
		\begin{tabular}{ccc}
			Messprobe & U/mV & Temperatur T$_\text{i}$/$^{\circ}\mathrm{C}$\\ \hline
			Wasser im Dewar-Gefäß bei Ausgangstemperatur, $T_x$& 0,7 & 17,52\\
			Mischwasser im Dewar-Gefäß bei Endtemperatur, $T_m$&1,9&47,11\\
			erhitzes Wasser, $T_y$&3,3&80,95
			% 10	&0{,}044\\
		\end{tabular}
		\caption{Messwerte und Temperaturen des Versuchsablaufs (Kalorimeter)}
		\label{taba1}
	\end{center}
\end{table} 	\begin{figure}[h]
		\begin{center}
		\includegraphics[scale=0.75]{picA.jpg}
		\caption{Grafisches Auftragen der Messwerte zur Zählrohr-Charakteristik}
		\label{picA}
		\end{center}	
	\end{figure}
\FloatBarrier
Die Länge des Plateau-Bereiches lässt sich darauf aufbauend auf ungefähr
290 Volt, von 360 Volt bis 650 Volt, abschätzen. Dieser Bereich ist in 
Abbildung \ref{picAlinregwitherrors} mit einem Messfehler von unter $1\%$ (\cite{anleitung}, Seite 226) 
noch einmal dargestellt, wobei in diesem Bereich (vgl. \ref{taba1})eine lineare 
Ausgleichsrechnung \cite{linreg} (Gl. \ref{eqlinrega}) programmiert in Python durchgeführt wurde:
\begin{align}
y&=a*x+b \\
\Leftrightarrow N&=0,0050\frac{1}{\text{s V}}*U+101,4606\frac{1}{\text{s}} \label{eqlinrega} \\
a&=0,0050\frac{1}{\text{s V}} \label{eqalinrega}\\
\Delta a&=38,0\% \\
b&=101,4606\frac{1}{\text{s}}\\
\Delta b&=0,9\%
\end{align}
	\begin{figure}[h]
		\begin{center}
		\includegraphics[scale=0.75]{picAlinregwitherrors.jpg}
		\caption{Plateau-Bereich der Zählrohr-Charakteristik}
		\label{picAlinregwitherrors}
		\end{center}	
	\end{figure}
\FloatBarrier
Aus der Linearen Regression lässt sich dann die Plateau-Steigung von $0,5\% \text{ pro } 
100 \text{ Volt }$(Gl. \ref{eqalinrega}) ablesen.
\subsection{Totzeitbestimmung nach der Zwei-Quellen-Methode}
\FloatBarrier
\begin{table}[h]
	\begin{center}
		\begin{tabular}{ccc}
			$N_1$[1/s]&$N_2$[1/s]&$N_{1+2}$[1/s] \\ \hline
			234,26&12,74&245,60\\
		\end{tabular}
		\caption{Zwei-Quellen-Methode}
		\label{tabc2}
	\end{center}
\end{table}
Die Totzeit lässt sich nach Gleichung \ref{2} aus der Zwei-Quellen-Methode mit Tabelle \ref{tabc2} berechnen,
der Fehler ergibt sich aus Gleichung \ref{totzeit2fehler}.
\begin{align}
\frac{N_{1+2}}{1 - T N_{1+2}}&=\frac{N_1}{1 - T N_1} + \frac{N_2}{1 - T N_2} \label{totzeit2fehler}\\
T= 234,5 \mu\text{s} \label{eqtot2}\\
\Delta T&=0,017\%
\end{align}
\subsection{Pro Teilchen freigesetzte Ladungsmenge im Zählrohr}
\FloatBarrier
Die pro Teilchen freigesetzte Ladungsmenge im Zählrohr lässt sich nach Gleichung \ref{ladungsmenge}
berechnen.
Daraus ergeben sich die Ladungsmengen in Tabelle \ref{tabd1} beziehungsweise Abbildung \ref{picD}.
In dem linearen Bereich, abgeschätzt von U$=360$V bis U$=650$V,wurde eine lineare Ausgleichsrechnung \cite{linreg} (Gl. \ref{eqlinregd}) 
programmiert in Python durchgeführt:
\begin{align}
y&=a*x+b \\
\Leftrightarrow Q/e&=(1,2093 * 10^8) \frac{1}{\text{V}} *U+ (-3,1548 * 10^10) \label{eqlinregd} \\
a&=(1,2093 * 10^8) \frac{1}{\text{V}} \label{eqalinregd}\\
\Delta a&=1,3\% \\
b&=-3,1548 * 10^10\\
\Delta b&=2,8\%
\end{align}
\begin{table}[h]
	\begin{center}
		\begin{tabular}{ccc}
			$U_{Heiz}$[V]&$I_{Heiz}$[A]&T[kK] \\ \hline
			% 10	&0,044\\
			4{,}200&2{,}100&1{,}959\\
			4{,}800&2{,}200&2{,}059\\
			5{,}000&2{,}300&2{,}108\\
			5{,}900&2{,}500&2{,}254\\
			6{,}100&2{,}600&2{,}298
		\end{tabular}
		\caption{Kathodentemperatur unter Heizleistungsvariation}
		\label{tabd1}
	\end{center}
\end{table} 	\begin{figure}[h]
		\begin{center}
		\includegraphics[scale=0.75]{picD.jpg}
		\caption{freigesetzte Ladungsmenge im Zählrohr}
		\label{picD}
		\end{center}	
	\end{figure}
\FloatBarrier
\section{Diskussion}
	\subsection{Eigenträgheitsmoment und Winkelrichtgröße}
Aus der Messung ergab sich die Winkelrichtgröße
$D=(5,642\pm1,247)*10^{-4}\frac{Fr}{^{\circ}}$


Das Eigenträgheitsmoment
$I_{D}=(-2,833\pm 0,013)*10^{-3}kg*m^2$
ist kleiner Null und wurde daher bei den nachfolgend Berechnungen vernachlässigt. Ein negativer Wert tritt auf, da das Trägheitsmoment des Stab, im Vergleich zur Drillachse, ein sehr hohes Trägheitsmoment hat. Kleine ungenauigkeiten bei der Messung haben eine große Auswirkung auf das Ergebins. Des weiteren gillt Gleichung \ref{eqti} nur für kleine Winkel, bei denen eine genause Messung der Periodendauer nicht möglich ist.
\\
\\
\subsection{Trägheitsmoment der Kugel und Scheibe}
Aus der Messung wurden die Trägheitsmomente
\begin{align*}
I_{Scheibe}=4,700*10{-5}kg*m^2\\
I_{Kugel}=4,739*10^{-5}kg*m^2
\end{align*}
bestimmt.


Die Berechnung nach Gleichung \ref{eqkugel} und \ref{eqzylsenk} ergab
\begin{align*}
I_{Scheibe}=2,532*10^{-3}\text{kg*m$^2$}\\
I_{Kugel}=6,944*10^{-2}\text{kg*m$^2$}
\end{align*}

\subsection{Trägheitsmoment der Trägheitspuppe}
Aus Gleichung \ref{eqkugel} bis \ref{eqzylpar}
\begin{align*}
I_{an}=(1,289\pm0,279)*10^{-3}\text{kg*m$^2$}\\
I_{an}=(1,397\pm0,280)*10^{-3}\text{kg*m$^2$}
\end{align*}
\\
\\
Die Berechnung aus der gemessenen Periodendauer ergibt Trägheitsmomente von
\begin{align*}
I_{an}=0,708*10^{-6}\text{kg*m2}\pm23,01\%\\
I_{ab}=1,112*10^{-6}\text{kg*m2}\pm22,14\%
\end{align*}
\\
\\
\subsection{Abweichung der gemessenen von den errechneten Werten}
Ins gesamt sind alle durch die Messung der Periodendauer ermittelten Werte $10{-3}$ mal kleiner als die errechneten. Das lässt darauf schließen, dass die ermittelte Winkelrichtgröße zu klein ist.
Fehler können durch ungenaues ablesen auf der lose aufgelegten Messskala oder ungenaue Kraftmesser entstanden sein.
% ========================================
%	Literaturverzeichnis
% ========================================

%\bibliographystyle{plainnat}			% Bibliographie-Style auswählen
%\bibliography{BIBDATEI}			% Literaturverzeichnis

% ========================================
%	Das Dokument endent
% ========================================
\includegraphics[scale=0.75]{img011.jpg}
\includegraphics[scale=0.75]{img012.jpg}
\end{document}