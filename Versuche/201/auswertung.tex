\subsection{Bestimmung der Wärmekapazität des Kaloriemeters}
\begin{table}[h]
	\begin{center}
		\begin{tabular}{cc}
			&Gewicht [g] \\ \hline
			% 10	&0{,}044\\
			Bleikörper&716,06\\
			Graphitkörper&135,87\\
			Kupferkörper&318,37\\
			Dewar-Gefäß, $m_D$&360,00
		\end{tabular}
		\caption{Massen der Versuchsgegenstände}
		\label{tabmassen}
	\end{center}
\end{table} \begin{table}[h]
	\begin{center}
		\begin{tabular}{ccc}
			Messprobe & U/mV & Temperatur T$_\text{i}$/$^{\circ}\mathrm{C}$\\ \hline
			Wasser im Dewar-Gefäß bei Ausgangstemperatur, $T_x$& 0,7 & 17,52\\
			Mischwasser im Dewar-Gefäß bei Endtemperatur, $T_m$&1,9&47,11\\
			erhitzes Wasser, $T_y$&3,3&80,95
			% 10	&0{,}044\\
		\end{tabular}
		\caption{Messwerte und Temperaturen des Versuchsablaufs (Kalorimeter)}
		\label{taba1}
	\end{center}
\end{table}
Die gemessenen Spannungen lassen sich mit Gleichung \ref{eqt} in Temperaturen umrechnen (Tab.\ref{taba1}).
Das Gewicht des hälftig aufgeteilten Wassers wurde auf $m_x=104,59\text{g}$ und $m_y=104,59\text{g}$ 
bestimmt. Daraus kann mit Gleichung \ref{eqcgmg} die Wärmekapazität des Kaloriemeters errechnet werden 
(Gl. \ref{eqcgmgval}).
\begin{align}
	c_gm_g&=62,65\frac{\text{J}}{\text{K}} \label{eqcgmgval}
\end{align}
\subsection{Bestimmung der Wärmekapazität von Graphit} \label{subsec:graphit}
\begin{table}[h]
	\begin{center}
		\begin{tabular}{c|cccc}
			Material&$\rho$ [g/cm$^3$]&$M$ [g/Mol]&$\alpha$ [10$^{-6}$ K$^{-1}$]&$\kappa [10^9\text{N}/\text{m}^2]$ \\ \hline
			Graphit&2,25&12,0&$\approx$8&33\\
			Kupfer&8,96&63,5&16,8&136\\
			Blei&11,35&207,2&29,0&42
		\end{tabular}
		\caption{Physikalische Eigenschaften der verwendeten Probematerialien [1]}
		\label{tablitdata}
	\end{center}
\end{table} \begin{table}[h]
	\begin{center}
		\begin{tabular}{ccc}
			Messprobe & U/mV & Temperatur T/$^{\circ}\mathrm{C}$\\ \hline
			Wasser im Dewar-Gefäß bei Ausgangstemperatur, $T_w$& 0,5 & 12,53\\
			Dewar-Gefäß-Inhalt bei Endtemperatur, $T_m$&0,9&22,49\\
			erhitze Probe, $T_k$&2,8&68,95
			% 10	&0{,}044\\
		\end{tabular}
		\caption{Messwerte und Temperaturen des Versuchsablaufs (Graphit)}
		\label{tabgraphit1}
	\end{center}
\end{table}
Die spezifische Wärmekapazität $c_k$ der Graphitprobe lässt sich mit Gleichung \ref{eqck} aus den Werten aus
Tabelle \ref{tabmassen} und Tabelle \ref{tabgraphit1}, sowie der Wassermenge im Dewargefäß von 
$m_w=599,00\text{g}-m_D=239,00\text{g}$ bestimmen (Gl. \ref{eqckgraphit}). Mit Gleichung \ref{eqckcp} lässt
sich daraus mit Tabelle \ref{tablitdata} die Atomwärme $C_P$ errechnen (Gl. \ref{eqcpgraphit}).
Der Zusammenhang zwischen $C_P$ und $C_V$ (Gl. \ref{eqcvcp}) führt zu Gleichung \ref{eqcpcv} (mit 
$V_0=M/\rho$) mit welcher sich $C_V$ für Graphit berechnen lässt (Gl. \ref{eqcvgraphit}).
\begin{align}
c_k&=1,67 \frac{\text{J}}{\text{K} \text{g}} \label{eqckgraphit}\\
C_P&=c_k * M \label{eqckcp} \\
C_P&=20,09 \frac{\text{J}}{\text{K} \text{Mol}} \label{eqcpgraphit} \\
C_V&= C_P - 9 \alpha^2 \kappa \frac{M}{\rho} T_m \label{eqcpcv}\\
C_V&= 20,06 \frac{\text{J}}{\text{K} \text{Mol}} \label{eqcvgraphit}
\end{align}
\subsection{Bestimmung der Wärmekapazität von Kupfer}
\begin{table}[h]
	\begin{center}
		\begin{tabular}{ccc}
			Messprobe & U/mV & Temperatur T/$^{\circ}\mathrm{C}$\\ \hline
			Wasser im Dewar-Gefäß bei Ausgangstemperatur, $T_w$& 0,5 & 12,53\\
			Dewar-Gefäß-Inhalt bei Endtemperatur, $T_m$&0,8&20,00\\
			erhitze Probe, $T_k$&3,0&73,76
			% 10	&0{,}044\\
		\end{tabular}
		\caption{Messwerte und Temperaturen des Versuchsablaufs (Kupfer)}
		\label{tabkupfer1}
	\end{center}
\end{table}
Die Wärmekapazitäten der Kupferprobe lassen sich mit Tabbelle \ref{tabkupfer1} und mit 
der Wassermenge im Dewargefäß von $m_w=571,99\text{g}-m_D=211,99\text{g}$ 
analog zu \ref{subsec:graphit} bestimmen.
\begin{align}
c_k&=0,41 \frac{\text{J}}{\text{K} \text{g}} \\
C_P&=26,31 \frac{\text{J}}{\text{K} \text{Mol}} \\
C_V&=25,59  \frac{\text{J}}{\text{K} \text{Mol}} 
\end{align}
\subsection{Bestimmung der Wärmekapazität von Blei} \label{subsec:blei}
\begin{table}[h]
	\begin{center}
		\begin{tabular}{ccc}
			Messprobe & U [mV] & Temperatur [$^{\circ}\mathrm{C}$]\\ \hline
			Wasser im Dewar-Gefäß bei Ausgangstemperatur, $T_w$&0,6&15,03\\
			Dewar-Gefäß-Inhalt bei Endtemperatur, $T_m$&0,8&20,00\\
			erhitze Probe, $T_k$&2,6&64,12
		\end{tabular}
		\caption{Messwerte und Temperaturen des Versuchsablaufs (Blei(1))}
		\label{tabblei1}
	\end{center}
\end{table} \begin{table}[h]
	\begin{center}
		\begin{tabular}{ccc}
			Messprobe & U [mV] & Temperatur [$^{\circ}\mathrm{C}$]\\ \hline
			Wasser im Dewar-Gefäß bei Ausgangstemperatur, $T_w$&0,6&15,03\\
			Dewar-Gefäß-Inhalt bei Endtemperatur, $T_m$&0,9&22,29\\
			erhitze Probe, $T_k$&2,9&71,36
		\end{tabular}
		\caption{Messwerte und Temperaturen des Versuchsablaufs (Blei(2))}
		\label{tabblei2}
	\end{center}
\end{table} \begin{table}[h]
	\begin{center}
		\begin{tabular}{ccc}
			Messprobe & U/mV & Temperatur T/$^{\circ}\mathrm{C}$\\ \hline
			Wasser im Dewar-Gefäß bei Ausgangstemperatur, $T_w$&0,6&15,03\\
			Dewar-Gefäß-Inhalt bei Endtemperatur, $T_m$&0,7&17,52\\
			erhitze Probe, $T_k$&3,0&73,76
		\end{tabular}
		\caption{Messwerte und Temperaturen des Versuchsablaufs (Blei(3))}
		\label{tabblei3}
	\end{center}
\end{table}
Die Wärmekapazitäten der Bleiprobe lassen sich aus den Tabbellen \ref{tabblei1}, \ref{tabblei2} und 
\ref{tabblei3} und mit den Wassermengen im Dewargefäß von $m_w=581,92\text{g}-m_D=221,92\text{g}$ 
analog zu \ref{subsec:graphit} bestimmen.
\begin{align}
\text{Bleiprobe (1)}\nonumber \\
c_{k,1}&=0,16 \frac{\text{J}}{\text{K} \text{g}} \\
C_{P,1}&=32,33 \frac{\text{J}}{\text{K} \text{Mol}} \\
C_{V,1}&=30,63  \frac{\text{J}}{\text{K} \text{Mol}} \\
\text{Bleiprobe (2)}\nonumber \\
c_{k,2}&=0,21 \frac{\text{J}}{\text{K} \text{g}} \\
C_{P,2}&=43,75 \frac{\text{J}}{\text{K} \text{Mol}} \\
C_{V,2}&=42,04  \frac{\text{J}}{\text{K} \text{Mol}} \\
\text{Bleiprobe (3)}\nonumber \\
c_{k,3}&=0,06 \frac{\text{J}}{\text{K} \text{g}} \\
C_{P,3}&=12,69 \frac{\text{J}}{\text{K} \text{Mol}} \\
C_{V,3}&=11,00  \frac{\text{J}}{\text{K} \text{Mol}} 
\end{align}


