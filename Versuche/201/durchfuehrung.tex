\subsection{Bestimmung der Wärmekapazität des Kaloriemeters}
Das Dewargefäß wurde leer und halb gefüllt mit destilliertem Wasser gewogen. Die andere Hälfte des Wassers wurde 
ebenfalls gewogen und dann über einer Kochplatte erhitzt. Die temperaturdifferenzabhängigen Spannungen des 
Wassers im Dewar-Gefäß und des erhitzen Wassers im Gegensatz zu einem mit Eiswasser gefüllten 
Behältnisses wurden mit einem Thermoelement bestimmt und direkt im Anschluss ist das erhitze Wasser
in das Dewar-Gefäß gefüllt worden. Nach einer kurzen Zeit des Vermischens wurde abermals die 
temperaturdifferenzabhängige Spannung mit dem Thermoelement gemessen.
\subsection{Bestimmung der Wärmekapazität der Proben}  
Bei den drei Proben von Blei, Graphit und Kupfer wurde jeweils analog vorgegangen, wobei Blei drei
mal den Versuch durchlief. Zuerst wurde die Masse der Proben bestimmt, die daraufhin in Wasser über 
der Kochplatte erhitzt worden sind.
Das Dewar-Gefäß wurde mit frischem destillierten Wasser befüllt gewogen, dann wurde die 
temperaturdifferenzabhängige Spannung des Wassers und der Probe gemessen. Die Probe wurde direkt 
im Anschluss komplett in das Wasser getaucht. Dann ist die temperaturdifferenzabhängige Spannung 
im Wasser und an der Probe solange durchgemessen worden, bis der gleiche Wert auftrat, 
welcher als mischtemperaturdifferenzabhängige Spannung aufgezeichnet wurde.