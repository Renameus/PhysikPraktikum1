\subsection{Bestimmung der Wärmekapazität des Kalorimeters}\label{subsec:diss1}
Bei der Bestimmung der Wassermassen ist durch die möglichst zu erreichende Halbierung des Wassers
vermutlich etwas ungenau gemessen worden, jedoch scheint das Ergebnis von 
$c_gm_g=62,65\frac{\text{J}}{\text{K}}$ ein akzeptabler Wert zu sein.\\
Wie auch bei den folgenden Messungen kommt es zu weiteren Ungenauigkeiten durch die Annahme,
dass keine Wärme an die Umgebung abgegeben wird. Da das in der Realität allerdings stattfindet, wird
das Ergebnis verfälscht.
\subsection{Bestimmung der Wärmekapazität verschiedener Proben}
\begin{table}[h]
 	\begin{center}
		\begin{tabular}{c|cccc}
			&Messung 1&Messung 2&Messung 3&Mittelwert \\ \hline
			$c_k [\text{J}/(\text{K } \text{g})]$&0,16&0,21&0,06&0,14$\pm$0,04\\
			$C_P [\text{J}/(\text{K } \text{Mol})]$&32,33&43,75&12,69&29,59$\pm$9,07\\
			$C_V [\text{J}/(\text{K } \text{Mol})]$&30,63&42,04&11,00&27,89$\pm$9,06
		\end{tabular}
		\caption{Wärmekapazitätenmittelung (Blei)}
		\label{tabdiss2}
	\end{center}
\end{table} \begin{table}[h]
	\begin{center}
		\begin{tabular}{cc}
		  	Probe&$C_V [\text{J}/(\text{K } \text{Mol})]$ \\ \hline
			Graphit&20,06\\
			Kupfer&25,59\\
			Blei&27,89
		\end{tabular}
		\caption{Wärmekapazitäten $C_V$ der Messproben}
		\label{tabdiss1}
	\end{center}
\end{table}
Werden die Werte aus \ref{subsec:blei} gemittelt (Gl. \ref{eqmittel}) und daraus der Fehler 
berechnet (Gl. \ref{eqmittelerr}), so lässt sich Tabelle \ref{tabdiss2} zusammenstellen.
\begin{align}
\overline x&=\frac{1}{N} \sum_{i=1}^N x_i \label{eqmittel} \\
S^2&=\frac{1}{N-1} \sum_{i=1}^N (x_i - \overline x)^2 \\
\Delta \overline x&=\frac{S}{\sqrt(N)} \label{eqmittelerr}
\end{align}
Werden die zusammengestellten Wärmekapazitäten (Tab. \ref{tabdiss1}) verglichen, so lässt
sich grundsätzlich eine Tendenz ablesen, je schwerer die Substanz, desto größer der Wert von
$C_V$, was auch grundsätzlich mit den theoretischen Annahmen übereinstimmt. Anders sieht das mit
den Vermutungen über ein Streben gegen $C_V=3R\approx24,94 \text{ }\frac{\text{J}}{\text{K} \text{Mol}}$
aus (für $R$ vgl. \cite{codatar}), der Mittelwert von Blei weicht sogar nach oben hin ab.\\
Dieses Verhalten lässt sich vermutlich durch verschiedenste Fehlerquellen erklären. Wie schon in
\ref{subsec:diss1} kommt es zu Ungenauigkeiten durch die Annahme, dass keine Wärme an die Umgebung 
abgegeben wird. Diese Problematik wird hier noch verstärkt durch das Abschätzen des Momentes, an welchem
$T_m$ erreicht wird. Eine weitere Unsicherheit bestand durch die mangelnde Genauigkeit (auf 0,1 mV) des 
Spannungsmessgerätes des Thermoelementes.  
\begin{table}[h]
	\begin{center}
		\begin{tabular}{cccc}
			Probe&$c_k/(\text{J}/(\text{K } \text{g}))$&$c_{k,lit}/(\text{J}/(\text{K } \text{g}))$&$\Delta c_k$ \\ \hline
			Graphit&1,67&0,71&135,2\%\\
			Kupfer&0,41&0,39&5,1\%\\
			Blei&$\approx$0,14&0,13&7,7\%
		\end{tabular}
		\caption{Literaturvergleich von $c_k$ [3]}
		\label{tadisslit}
	\end{center}
\end{table}
Vergleicht man die Literaturangaben \cite{tafel} $c_{k,lit}$ mit den in dem Versuch ermittelten Werten,
so ist eine recht gute Annährung für Blei und Kupfer zu erkennen, $c_k$ von Graphit weicht allerdings stark
von dem Literaturwert ab. Außerdem ist auffällig, dass alle ermittelten Werte nach oben hin abweichen.
\FloatBarrier