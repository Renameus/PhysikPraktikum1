\subsection{Wärmekapazität}
Wärmekapazitäten $c$ beschreiben das Vermögen eines Körpers bei einer Temperaturänderung
Wärme zu speichern. Dabei wird unterteilt in $C_P$, Wärmekapazität bei konstantem Druck, und 
$C_V$, Wärmekapazität bei konstantem Volumen. Das Dulong-Petitsche Gesetz \cite{anleitung} 
(Gl. \ref{eqdulpet}) beruht auf der klassischen Physik und beschreibt die Wärmekapazität auf 
Basis der Schwingungen von Atomen mit beliebiger Energie. $R$ ist hier die allgemeine Gaskonstante.
\begin{align}
C_V&=3 R \label{eqdulpet}
\end{align}
Dieses Gesetz ist bei schwereren Atomen bei höherer Temperatur auch näherungsweise gültig, jedoch
ist mit der Quantenmechanik bekannt, dass die Schwingungsenergie nur gequantelt auftritt, also 
keine beliebigen Werte annehmen kann. Aufgrund dieser Beschaffenheit lässt sich schließen, dass
die mittlere kinetische Energie eines Körpers immer kleiner ist, als man nach Dulong-Petit folgern 
würde.\\
\subsection{Das Mischungskalorimeter}
Die Funktionsweise des Mischungskalorimeters beruht auf der Temperaturänderung eines mit Wasser 
bekannter Masse und Temperatur gefüllten Dewar-Gefäßes bei dem Hinzufügen der erhitzen Probe 
bekannter Masse und Temperatur bei gleichbleibendem Druck. Ist die Wärmekapazität $c_g m_g$ 
(Gl. \ref{eqcgmg},\cite{anleitung} ) des Kalorimeters bekannt, kann mit Gleichung  (\ref{eqck}) \cite{anleitung}
die Wärmekapazität $c_k$ der Probe berechnet werden. Dabei ist $C_w$ die Wärmekapazität von Wasser, 
$m_x$ und $m_y$ sind die Wassermassen des Wassers mit den Temperaturen $T_x$ und $T_y$, welche zur 
Bestimmmung von $c_gm_g$ die Mischtemperatur $T_m'$ ergeben, wenn die beiden Teile des Wasser im 
Dewar-Gefäß vermischt sind. Ähnlich ist die Größe $T_m$ zu verstehen, welche die Mischtemperatur 
der Massen $m_w$, der Wassermasse, und $m_k$, der Masse der erhitzen Probe, mit den Temperaturen $T_w$,
 der Temperatur des Wassers, und $T_k$, der Temperatur der Probe, beschreibt.
\begin{align}
c_gm_g&=\frac{c_w m_y (T_y - T_m')-c_w m_x(T_y-T_m')}{T_m'-T_x} \label{eqcgmg} \\
c_k&=\frac{(c_w m_w + c_g m_g)(T_m-T_w)}{m_k(T_k - T_m)} \label{eqck}
\end{align}
Das sich aus der bekannten molaren Masse der Probe ergebende $C_P$ lässt sich dann mit
Gleichung  (\ref{eqcvcp}) \cite{anleitung}  in $C_V$ umrechnen. $\alpha$ ist der lineare 
Ausdehnungskoeffizient der Probe, $\kappa$ das Kompressionsmodul der Probe, $V_0$ das Molvolumen
der Probe und $T$ die zugehörige Temperatur.
\begin{align}
C_P-C_V&=9 \alpha^2 \kappa V_0 T \label{eqcvcp}
\end{align}
\subsection{Das Thermoelement}
Das Thermoelement dient der Temperaturdifferenzmessung. Der bei Temperaturdifferenzen auftretende
Elektronendrift beeinflusst das Kontaktpotential zwischen zwei verschiedenen Metallen, sodass sich 
aus der auftretenden Spannung zwischen zwei solcher Kontaktstellen auf die Temperaturdifferenz schließen lässt \cite{anleitung}.
\begin{align}
T&=25,157\text{ }U_{th} - 0,19\text{ } U_{th}^2 \text{ }(U_{th} \text{ in mV}) \label{eqt}\\
\text{Gültigkeitsbereich:  \quad} 0<T<100^\circ \text{ C}\nonumber
\end{align}