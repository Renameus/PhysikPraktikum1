\subsection{spezifische Wärmekapazität}
Spezifische Wärmekapazitäten $C$ beschreiben das Vermögen eines Körpers bei einer Temperaturänderung
Wärme zu speichern. Dabei wird unterteilt in $C_P$, spezifische Wärmekapazität bei konstantem Druck, und 
$C_V$ spezifische Wärmekapazität bei konstantem Volumen. Das Dulong-Petitsche Gesetz \cite{anleitung} 
(Gl. \ref{eqdulpet}) beruht auf der klassischen Physik und beschreibt die Wärmekapazität auf 
Basis der Schwingungen von Atomen mit beliebiger Energie.
\begin{align}
C_V&=3 R \label{eqdulpet}
\end{align}
Dieses Gesetz ist bei schwereren Atomen bei höherer Temperatur auch näherungsweise gültig, jedoch
ist mit der Quantenmechanik bekannt, dass die Schwingungsenergie nur gequantelt auftritt, also 
keine beliebigen Werte annehmen kann. Aufgrund dieser Beschaffenheit lässt sich schließen, dass
die mittlere kinetische Energie eines Körpers immer kleiner ist als man nach Dulong-Petit folgern 
würde.\\
\subsection{Das Mischungskaloriemeter}
Die Funktionsweise des Mischungskaloriemeters beruht auf der Temperaturänderung eines mit Wasser 
bekannter Masse und Temperatur gefüllten Dewar-Gefässes bei dem Hinzufügen der erhitzen Probe 
bekannter Masse und Temperatur bei gleichbleibendem Druck. Ist die Wärmekapazität $c_g m_g$ 
(Gl. \ref{eqcgmg},\cite{anleitung} ) des Kaloriemeters bekannt, kann mit Gleichung  \ref{eqck} 
die spezifische \cite{anleitung} Wärmekapazität $c_k$ der Probe berechnet werden. 
\begin{align}
c_gm_g&=\frac{c_w m_y (T_y - T_m')-c_w m_x(T_y-T_m')}{T_m'-T_x} \label{eqcgmg} \\
c_k&=\frac{(c_w m_w + c_g m_g)(T_m-T_w)}{m_k(T_k - T_m)} \label{eqck}
\end{align}
Das sich aus der bekannten molaren Masse der Probe ergebende $C_P$ lässt sich dann mit
Gleichung  \ref{eqcvcp} \cite{anleitung}  in $C_V$ umrechnen.
\begin{align}
C_P-C_V&=9 \alpha^2 \kappa V_0 T \label{eqcvcp}
\end{align}
\subsection{Das Thermoelement}
Das Thermoelement dient der Temperaturdifferenzmessung. Der bei Temperaturdifferenzen auftretende
Elektronendrift beeinflusst das Kontaktpotential zwischen zwei verschiedenen Metallen, sodass sich 
aus der auftretenden Spannung auf die Temperaturdifferenz schließen lässt \cite{anleitung}.
\begin{align}
T&=25,157 U_{th} - 0,19 U_{th}^2 (U_{th} \text{in mV}) \label{eqt}
\end{align}