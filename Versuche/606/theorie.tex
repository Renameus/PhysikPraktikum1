Eine wichtige Größe des Magnetismus ist die Suszeptibilität $\chi$, welche die Magnetisierbarkeit 
eines Stoffes in einem Magnetfeld angibt.\\ 
Im Gegensatz zu diamagnetischen Substanzen, deren induzierte magnetische Momente dem Magnetfeld 
entgegengerichtet sind, ist bei paramagnetischen Substanzen die Suszeptibilität positiv.
Diese Eigenschaft hat den Ursprung in einem nicht verschwindenden Drehimpuls, welcher sich aus
den Drehimpulsen der Elektronen, der Elektronenhülle und des Kerns zusammensetzt. Dieser sorgt dafür,
dass sich die magnetischen Momente relativ zum angelegten magnetischen Feld ausrichten. Da diese 
Orientierung durch Bewegung der atomaren Objekte gestört wird, besteht eine Temperaturabhängigkeit
der Suszeptibilität. Im Allgemeinen ist diese nicht trivial von der Temperatur und der magnetischen 
Feldstärke abhängig. Eine konkrete Gleichung für die Berechnung der Suszeptibilität lässt sich
aus der Gleichung für die Magnetisierung unter Zuhilfenahme der Quantenphysik mit Benutzung des 
Landé-Faktors, des Zeeman-Effektes und der Brillouifunktion herleiten. Bei Hochtemperatur lässt
sich $\chi$ passend zum Curieschen Gesetz wie folgt nähern.
\begin{align}
\chi&=\frac{\mu_0 \mu_B^2 g_J^2 N J(J+1)}{3 k T} \label{eqn:curie}\\
\chi&\sim \frac{1}{T}
\end{align}
\\
N ist die Anzahl an magnetischen Momenten pro Volumeneinheit. Sie lässt sich aus der Dichte errechnen. $M$ ist die Molmasse und $N_A$ die Avogadro-Konstante.
\begin{align}
N=\frac{\rho}{M}*N_A*Z
\end{align}

Starken Paramagnetismus weisen Ionen Seltener Erden auf, diese besitzen innere 4f Elektronen, also 
eine Elektronenhülle mit großen Drehimpulsen. Der Gesamtdrehimpuls $\vec J$ ist dabei durch die
Hundschen Regeln mit dem Pauli-Prinzip \cite{anleitung} festgelegt:
	\begin{table}[h]
	%\begin{description}
			\begin{enumerate}
				\item Gesamtspin $\vec S=\sum \vec s_i$ 
				\item maximaler Drehimpuls $\vec L=\sum \vec l_i$ 
				\item Gesamtdrehimpuls 
				\begin{itemize}
					\item Schale weniger als halbvoll $\vec J=\vec L - \vec S$
					\item Schale mehr als halbvoll $\vec J=\vec L + \vec S$
				\end{itemize}
			\end{enumerate}
	%\end{description}
	\label{hund}
	\end{table}
\FloatBarrier