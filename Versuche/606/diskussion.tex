\subsection{Güte}
Für die Güte Bandfilters ergab sich
\begin{align*}
Q=70
\end{align*}
Der gemessene Wert unterschreitet sich von Geräteangabe von Q=100 deutlich. In Abbildung \ref{gute} ist zu erkennen, dass zwei der Messwerte nah an $\nu_-$ und $\nu_+$ sind. $\nu_0$ befindet sich nach sowohl nach der Messung, als auch nach der Geräteangabe bei 35kHz. Es ist also auszuschließen, dass das Ergebnis durch eine zu geringe Messwerteanzahl verfälscht wurde.

\subsection{Suszeptibilität seltener Erden}
Errechnete Suszeptiblitäten:
\begin{align*}
&\chi(Nd_2O_3)=3,02*10^{-3}&&\chi(Gd_2O_3)=13,79*10^{-3}&&\chi(Dy_2O_3)=25,41*10^{-3}&
\end{align*}
\\
Gemessen durch Brückenspannung:
\begin{align*}
&\chi(Nd_2O_3)=(4,17\pm0,27)*10^{-3}&&\chi(Gd_2O_3)=(21,01\pm0,39)*10^{-3}&\\&\chi(Dy_2O_3)=(45,07\pm0,53)*10^{-3}&
\end{align*}
\\
Gemessen durch Widerstandsvariaton:
\begin{align*}
&\chi(Nd_2O_3)=(2,73\pm0,34)*10^{-3}&&\chi(Gd_2O_3)=(16,52\pm0,39)*10^{-3}&\\&\chi(Dy_2O_3)=(36,19\pm0,21)*10^{-3}&
\end{align*}

Vergleicht man nun die Suzeptibilitätswerte der verschiedenen Proben, so ist zwar ein signifikanter Unterschied
zu bemerken, jedoch fällt auf, dass zumindest die Größenordnung übereinstimmt. Bei der Messwerablesung waren die 
Werte teilweise schwierig abzugrenzen, was natürlich zu einer gewissen Ungenauigkeit führt. \\Eine weitere Auffälligkeit ist
bei dem Vergleich der beiden Messverfahren zu sehen. $\chi_U$ ist immer größer als $\chi_R$. Eine mögliche
Quelle dieser Abweichung könnte $U_{Sp}$ sein, welches einen größeren Wert besaß als angenommen. Da das 
Widerstandsvariationsmessverfahren insgesamt näher an den Theoriewerten liegt, ist anzunehmen, dass dieses
Verfahren genauer ist oder sich zumindest besser zum Messen eignet. \\Bis auf
$\chi(R,Nd_2O_3)$ sind außerdem alle Versuchswerte größer als nach der Theorie angenommen wurde.